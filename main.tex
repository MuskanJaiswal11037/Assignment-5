
\usepackage[utf8]{inputenc}
\documentclass{beamer}
\usetheme{CambridgeUS}
\usepackage{listings}
\usepackage{blkarray}
\usepackage{listings}
\usepackage{subcaption}
\usepackage{url}
\usepackage{tikz}
\usepackage{tkz-euclide} % loads  TikZ and tkz-base
%\usetkzobj{all}
\usetikzlibrary{calc,math}
\usepackage{float}
\renewcommand{\vec}[1]{\mathbf{#1}}
\usepackage[export]{adjustbox}
\usepackage[utf8]{inputenc}
\usepackage{amsmath}
\usepackage{amsfonts}
\usepackage{tikz}
\usepackage{hyperref}
\usepackage{bm}
\usetikzlibrary{automata, positioning}
\providecommand{\pr}[1]{\ensuremath{\Pr\left(#1\right)}}
\providecommand{\mbf}{\mathbf}
\providecommand{\qfunc}[1]{\ensuremath{Q\left(#1\right)}}
\providecommand{\sbrak}[1]{\ensuremath{{}\left[#1\right]}}
\providecommand{\lsbrak}[1]{\ensuremath{{}\left[#1\right.}}
\providecommand{\rsbrak}[1]{\ensuremath{{}\left.#1\right]}}
\providecommand{\brak}[1]{\ensuremath{\left(#1\right)}}
\providecommand{\lbrak}[1]{\ensuremath{\left(#1\right.}}
\providecommand{\rbrak}[1]{\ensuremath{\left.#1\right)}}
\providecommand{\cbrak}[1]{\ensuremath{\left\{#1\right\}}}
\providecommand{\lcbrak}[1]{\ensuremath{\left\{#1\right.}}
\providecommand{\rcbrak}[1]{\ensuremath{\left.#1\right\}}}
\providecommand{\abs}[1]{\vert#1\vert}

\newcounter{saveenumi}
\newcommand{\seti}{\setcounter{saveenumi}{\value{enumi}}}
\newcommand{\conti}{\setcounter{enumi}{\value{saveenumi}}}
\usepackage{amsmath}
\setbeamertemplate{caption}[numbered]{}

\title{\typedef{ASSIGNMENT 5}}       
\author{MUSKAN JAISWAL -cs21btech11037}
\date{May 2022}
\logo{\large \Latex{}}
\begin{document}
\begin{frame}
		\titlepage
	\end{frame}

\begin{frame}{Outline}
  \tableofcontents
\end{frame}

\section{Abstract}
	\begin{frame}{Abstract}
		\begin{itemize}
			\item 	This document contains the explanation of example 4.5 of Papoulis Pillai Probability book
		\end{itemize}
	\end{frame}


\maketitle

\section{QUESTION:}
\begin{frame}{}
\begin{block}{}
A telephone call occurs at random in the interval (0,1).In this experiment, the outcomes are time distances t between 0 and 1 and the probability that t is between t1 and t2 is given by \bigskip

P\{t1 \le t \le t2\}=t2-t1 \\
Find the C.D.F of the given event.
\end{block}
\end{frame}
\section{ANSWER:}
\begin{frame}
Let the  random variable x such that x(t)=t  $ 0 \le t \le1.$
Here, t is the outcome of the experiment and also the corresponding value x(t) of the random variable x.\\
If x $>$ 1, then X(t) $\le$ x for every outcome. Hence
$F(x) = P\{X \le x\}=P\{0 \le t \le 1\}=P(S)=1$ \\

If $0 \le x \le 1$, then X(t) $\le$ x, for t in (0,x). Hence
$F(x)=P\{X \le x\}=P\{0 \le t \le x\}=x \\
If x $<$ 0, then \{X \le x\} $ is the impossible event because $ x(t) \ge 0 ,as 0 \le t \le 1$
Hence, $F(x)=P\{X \le x\}=P\{\phi\}=0 $ \\
\end{frame}

\end{document}
